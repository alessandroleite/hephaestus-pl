\section{Related Work}
\label{related-work}

%@vra: to be refined and extended

Previous work discusses the need to address SPL tool development as SPL development itself~\citep{grunbacher:2008}. Nevertheless, detailed guidelines on domain and application engineering were not explored.

As discussed in Section~\ref{sec:domainAnalysis}, a key requirement of \hpl{} is to support or-configurability of product artifacts, which has not been explicitly addressed elsewhere.

In \citep{batory-ahead-bootstrap} was bootstrapped AHEAD from AHEAD. This is similar to our work, but they do not focus on addressing different artifacts and no explicit support for or-feature is provided. Furthermore, the programming language and paradigm is different from ours.

Transkell~\citep{marcos:2010} is a domain specific language (DSL) developed to extract and modularize the variabilities of Hephaestus tool and makes it a product line. It is a technique of transformation approach similar to our work.
%Thus, it offers flexibility to extend the tool to new domains. 
The Hephaestus’s feature model represents the transformations of Hephaestus and the syntax and semantics of the Transkell language were implemented in the environment Stratego/XT~\citep{visser:2003}. In our work, we focus on assets variabilities that represent the different domain artifacts of \hp{} and we present the details of \hpl{} developed as a product line. 

The \textit{DOPLER (Decision-Oriented Product Line Engineering for effective Reuse)} approach~\citep{DBLP:journals/ase/DhunganaGR11} represents a tool suite to SPL tool development as SPL development itself~\citep{grunbacher:2008}. It comprises \textit{DoplerVML}, a variability modeling language to define product lines based decision models with emphasis on the derivation of products.
\textit{DOPLER} was initially developed to support the domain of industrial automation (Siemens VAI), but the proposal to be extensible and customizable to different contexts to meet the needs of different users and organizations.
However, the approach \textit{DOPLER}  does not support configurability of any combination of assets as we propose in the solution of our work.

Some other comparative studies with product derivation tools were conducted ~\citep{DBLP:conf/gpce/TorresKSBTBCLBM10, uiraWBDDM:2010}. 
~\citep{DBLP:conf/gpce/TorresKSBTBCLBM10} has analyzed six modern product derivation tools (Captor, CIDE, GenArch, Hephaestus, pure::variants e XVCL) in the context of evolution scenarios of a software product line. The study analyzed the modularity, complexity and stability of product derivation artifacts along  evolution of a mobile product line. The evaluation showed that  modularity and  stability requirements in software product lines are favorable to the flexibility of the tool.
In ~\citep{uiraWBDDM:2010} the flexibility and extensibility of product line development tools that support product derivation based on DDM and AOSD approaches were evaluated to support the addition of new features.

TODO: Talk about \cite{deltaSchaefer} in related work