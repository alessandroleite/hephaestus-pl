\section{Conclusion}
\label{conclusion}

Hephaestus-PL is a product line resulting from evolution of the Hephaestus tool. 
Hephaestus~\cite{rbonifacio:sbcars2009} was initially designed to support application engineering and variability management in a product line of use case scenarios.
%To manage the variability in use case scenarios, Hephaestus uses a compositional approach, in particular, aspect-orientation, that combines four input artifacts of the product line (SPL Model, Feature Model, Product Configuration and Configuration Knowledge) to the derivation of product instances.
Over time, Hephaestus evolved to manage variability in other assets such as requirements, source code, and business processes. However, this tool was not designed with flexibility and configurability in mind to allow its customization to address variability in a new specific asset nor in any desired combination of assets.

To address this shortcoming, Hephaestus-PL was developed. 
%It is a software product line that was extracted from different versions of Hephaestus, each version being a tool addressing variability in a number of artifacts.
Hephaestus-PL increases the configurability of Hephaestus by allowing the derivation of instance tools managing variability in any combination of artifacts; additionally, its flexibility allows for further systematic extension to add new assets and their combination. 
%Since Hephaestus-PL was bootstrapped from such independent versions of Hephaestus, \hpl{}'s CK, as well as \hp{}'s CK, represents a mapping of feature expressions to transformations.
%Hephaestus-PL implements six high-level transformations to generate an instance of Hephaestus, namely SelectBaseProduct, SelectAsset, SelectExport, BindProductName, RemoveProductMainFunction and SelectCKParser. 
%These transformations of Hephaestus-PL’s CK are built from a combination of set of meta-programming operations that when executed perform modifications in the product Hephaestus instance being generated.
Once Hephaestus-PL was bootstrapped, we defined a reactive approach to increase its configurability and to reach the goal of enabling the generation of different instances of Hephaestus.

%--talk about assessment
An assessment reveals that \hpl{} has improved configurability and flexibility when compared to previous evolution of \hp.
The proposed \hpl{}'s architecture and the approach of variability management used in \hpl{} address configurability and flexibility in the development of SPL derivation tools. Thus, we using a transformational approach with metaprogramming operations to extend the variation points of the base product of \hpl{} instance. 

In \hpl{} some degree of modularity was obtained by the mapping of \hpl's product configurations to metaprogramming transformations where the \hpl's variability related to the assets' configuration is handled dynamically and automatically in the generation of a \hpl{} variant. This is allowed by the transformational process of generating products in \hpl{} with the support of the metaprogramming operations that meet the needs of managing the variability of assets in the artifacts (open data types and open functions) of the base module of \hpl.
Besides, the reactive process defined in \hpl{} to introduce support for managing variabilities in new assets and it contributes to the flexibility of \hpl{}. 

%--restrictions (limitations) about our work
Although our study focuses on a single tool, we believe that its design and supporting reactive process could be used to improve configurability and flexibility in other SPL product derivation tools.
Nevertheless, further empirical work is necessary to address the external validity threat. 
We also plan to conduct further empirical studies assessing Hephaestus-PL's evolution to handle variability in diferent kind of artifacts.

%--future works
As future work, we propose the definition of \textit{Design Rules} that represent a mechanism that would allow reduction of the size of the asset metadata structures in \hpl. Using an inference process into the asset modules, especially the modules that define the data types and the transformations of the asset, it would be possible to exctract information, currently contained in the asset metadata structures, to extend the variability points of a \hpl{} base product. In this case, we could reduce the size of the asset metadata structures.  Another intermediate solution that would bring a good reduction of the asset metadata structures would work with two pieces of information, a \textit{acronym} and a \textit{name} of asset, and deriving most of the other pieces to extend the variation points of \hpl{}.

Formally ensuring well-formedness of the generated products is outside
the scope of this work. Recent work has addressed this for different
variability mechanisms.
For instance,  CFJ~\cite{kaestner-tosem12} addresses type checking
for annotative variability  mechanism,  FFJ~\cite{apel-ase08}
addresses the same issue in a compositional mechanism, whereas
DeltaJ~\cite{schaefer-aosd09} does in  a transformational approach.
The latter approach supports exactly three kinds of transformations,
addition or removal of classes or modification of  classes by adding
or removing methods and fields or by wrapping methods. We hypothesise
that such formalism could inspire formal work showing type safety for
Hephaestus-PL.